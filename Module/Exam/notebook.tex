
% Default to the notebook output style

    


% Inherit from the specified cell style.




    
\documentclass[11pt]{article}

    
    
    \usepackage[T1]{fontenc}
    % Nicer default font (+ math font) than Computer Modern for most use cases
    \usepackage{mathpazo}

    % Basic figure setup, for now with no caption control since it's done
    % automatically by Pandoc (which extracts ![](path) syntax from Markdown).
    \usepackage{graphicx}
    % We will generate all images so they have a width \maxwidth. This means
    % that they will get their normal width if they fit onto the page, but
    % are scaled down if they would overflow the margins.
    \makeatletter
    \def\maxwidth{\ifdim\Gin@nat@width>\linewidth\linewidth
    \else\Gin@nat@width\fi}
    \makeatother
    \let\Oldincludegraphics\includegraphics
    % Set max figure width to be 80% of text width, for now hardcoded.
    \renewcommand{\includegraphics}[1]{\Oldincludegraphics[width=.8\maxwidth]{#1}}
    % Ensure that by default, figures have no caption (until we provide a
    % proper Figure object with a Caption API and a way to capture that
    % in the conversion process - todo).
    \usepackage{caption}
    \DeclareCaptionLabelFormat{nolabel}{}
    \captionsetup{labelformat=nolabel}

    \usepackage{adjustbox} % Used to constrain images to a maximum size 
    \usepackage{xcolor} % Allow colors to be defined
    \usepackage{enumerate} % Needed for markdown enumerations to work
    \usepackage{geometry} % Used to adjust the document margins
    \usepackage{amsmath} % Equations
    \usepackage{amssymb} % Equations
    \usepackage{textcomp} % defines textquotesingle
    % Hack from http://tex.stackexchange.com/a/47451/13684:
    \AtBeginDocument{%
        \def\PYZsq{\textquotesingle}% Upright quotes in Pygmentized code
    }
    \usepackage{upquote} % Upright quotes for verbatim code
    \usepackage{eurosym} % defines \euro
    \usepackage[mathletters]{ucs} % Extended unicode (utf-8) support
    \usepackage[utf8x]{inputenc} % Allow utf-8 characters in the tex document
    \usepackage{fancyvrb} % verbatim replacement that allows latex
    \usepackage{grffile} % extends the file name processing of package graphics 
                         % to support a larger range 
    % The hyperref package gives us a pdf with properly built
    % internal navigation ('pdf bookmarks' for the table of contents,
    % internal cross-reference links, web links for URLs, etc.)
    \usepackage{hyperref}
    \usepackage{longtable} % longtable support required by pandoc >1.10
    \usepackage{booktabs}  % table support for pandoc > 1.12.2
    \usepackage[inline]{enumitem} % IRkernel/repr support (it uses the enumerate* environment)
    \usepackage[normalem]{ulem} % ulem is needed to support strikethroughs (\sout)
                                % normalem makes italics be italics, not underlines
    

    
    
    % Colors for the hyperref package
    \definecolor{urlcolor}{rgb}{0,.145,.698}
    \definecolor{linkcolor}{rgb}{.71,0.21,0.01}
    \definecolor{citecolor}{rgb}{.12,.54,.11}

    % ANSI colors
    \definecolor{ansi-black}{HTML}{3E424D}
    \definecolor{ansi-black-intense}{HTML}{282C36}
    \definecolor{ansi-red}{HTML}{E75C58}
    \definecolor{ansi-red-intense}{HTML}{B22B31}
    \definecolor{ansi-green}{HTML}{00A250}
    \definecolor{ansi-green-intense}{HTML}{007427}
    \definecolor{ansi-yellow}{HTML}{DDB62B}
    \definecolor{ansi-yellow-intense}{HTML}{B27D12}
    \definecolor{ansi-blue}{HTML}{208FFB}
    \definecolor{ansi-blue-intense}{HTML}{0065CA}
    \definecolor{ansi-magenta}{HTML}{D160C4}
    \definecolor{ansi-magenta-intense}{HTML}{A03196}
    \definecolor{ansi-cyan}{HTML}{60C6C8}
    \definecolor{ansi-cyan-intense}{HTML}{258F8F}
    \definecolor{ansi-white}{HTML}{C5C1B4}
    \definecolor{ansi-white-intense}{HTML}{A1A6B2}

    % commands and environments needed by pandoc snippets
    % extracted from the output of `pandoc -s`
    \providecommand{\tightlist}{%
      \setlength{\itemsep}{0pt}\setlength{\parskip}{0pt}}
    \DefineVerbatimEnvironment{Highlighting}{Verbatim}{commandchars=\\\{\}}
    % Add ',fontsize=\small' for more characters per line
    \newenvironment{Shaded}{}{}
    \newcommand{\KeywordTok}[1]{\textcolor[rgb]{0.00,0.44,0.13}{\textbf{{#1}}}}
    \newcommand{\DataTypeTok}[1]{\textcolor[rgb]{0.56,0.13,0.00}{{#1}}}
    \newcommand{\DecValTok}[1]{\textcolor[rgb]{0.25,0.63,0.44}{{#1}}}
    \newcommand{\BaseNTok}[1]{\textcolor[rgb]{0.25,0.63,0.44}{{#1}}}
    \newcommand{\FloatTok}[1]{\textcolor[rgb]{0.25,0.63,0.44}{{#1}}}
    \newcommand{\CharTok}[1]{\textcolor[rgb]{0.25,0.44,0.63}{{#1}}}
    \newcommand{\StringTok}[1]{\textcolor[rgb]{0.25,0.44,0.63}{{#1}}}
    \newcommand{\CommentTok}[1]{\textcolor[rgb]{0.38,0.63,0.69}{\textit{{#1}}}}
    \newcommand{\OtherTok}[1]{\textcolor[rgb]{0.00,0.44,0.13}{{#1}}}
    \newcommand{\AlertTok}[1]{\textcolor[rgb]{1.00,0.00,0.00}{\textbf{{#1}}}}
    \newcommand{\FunctionTok}[1]{\textcolor[rgb]{0.02,0.16,0.49}{{#1}}}
    \newcommand{\RegionMarkerTok}[1]{{#1}}
    \newcommand{\ErrorTok}[1]{\textcolor[rgb]{1.00,0.00,0.00}{\textbf{{#1}}}}
    \newcommand{\NormalTok}[1]{{#1}}
    
    % Additional commands for more recent versions of Pandoc
    \newcommand{\ConstantTok}[1]{\textcolor[rgb]{0.53,0.00,0.00}{{#1}}}
    \newcommand{\SpecialCharTok}[1]{\textcolor[rgb]{0.25,0.44,0.63}{{#1}}}
    \newcommand{\VerbatimStringTok}[1]{\textcolor[rgb]{0.25,0.44,0.63}{{#1}}}
    \newcommand{\SpecialStringTok}[1]{\textcolor[rgb]{0.73,0.40,0.53}{{#1}}}
    \newcommand{\ImportTok}[1]{{#1}}
    \newcommand{\DocumentationTok}[1]{\textcolor[rgb]{0.73,0.13,0.13}{\textit{{#1}}}}
    \newcommand{\AnnotationTok}[1]{\textcolor[rgb]{0.38,0.63,0.69}{\textbf{\textit{{#1}}}}}
    \newcommand{\CommentVarTok}[1]{\textcolor[rgb]{0.38,0.63,0.69}{\textbf{\textit{{#1}}}}}
    \newcommand{\VariableTok}[1]{\textcolor[rgb]{0.10,0.09,0.49}{{#1}}}
    \newcommand{\ControlFlowTok}[1]{\textcolor[rgb]{0.00,0.44,0.13}{\textbf{{#1}}}}
    \newcommand{\OperatorTok}[1]{\textcolor[rgb]{0.40,0.40,0.40}{{#1}}}
    \newcommand{\BuiltInTok}[1]{{#1}}
    \newcommand{\ExtensionTok}[1]{{#1}}
    \newcommand{\PreprocessorTok}[1]{\textcolor[rgb]{0.74,0.48,0.00}{{#1}}}
    \newcommand{\AttributeTok}[1]{\textcolor[rgb]{0.49,0.56,0.16}{{#1}}}
    \newcommand{\InformationTok}[1]{\textcolor[rgb]{0.38,0.63,0.69}{\textbf{\textit{{#1}}}}}
    \newcommand{\WarningTok}[1]{\textcolor[rgb]{0.38,0.63,0.69}{\textbf{\textit{{#1}}}}}
    
    
    % Define a nice break command that doesn't care if a line doesn't already
    % exist.
    \def\br{\hspace*{\fill} \\* }
    % Math Jax compatability definitions
    \def\gt{>}
    \def\lt{<}
    % Document parameters
    \title{Exam 1}
    
    
    

    % Pygments definitions
    
\makeatletter
\def\PY@reset{\let\PY@it=\relax \let\PY@bf=\relax%
    \let\PY@ul=\relax \let\PY@tc=\relax%
    \let\PY@bc=\relax \let\PY@ff=\relax}
\def\PY@tok#1{\csname PY@tok@#1\endcsname}
\def\PY@toks#1+{\ifx\relax#1\empty\else%
    \PY@tok{#1}\expandafter\PY@toks\fi}
\def\PY@do#1{\PY@bc{\PY@tc{\PY@ul{%
    \PY@it{\PY@bf{\PY@ff{#1}}}}}}}
\def\PY#1#2{\PY@reset\PY@toks#1+\relax+\PY@do{#2}}

\expandafter\def\csname PY@tok@w\endcsname{\def\PY@tc##1{\textcolor[rgb]{0.73,0.73,0.73}{##1}}}
\expandafter\def\csname PY@tok@c\endcsname{\let\PY@it=\textit\def\PY@tc##1{\textcolor[rgb]{0.25,0.50,0.50}{##1}}}
\expandafter\def\csname PY@tok@cp\endcsname{\def\PY@tc##1{\textcolor[rgb]{0.74,0.48,0.00}{##1}}}
\expandafter\def\csname PY@tok@k\endcsname{\let\PY@bf=\textbf\def\PY@tc##1{\textcolor[rgb]{0.00,0.50,0.00}{##1}}}
\expandafter\def\csname PY@tok@kp\endcsname{\def\PY@tc##1{\textcolor[rgb]{0.00,0.50,0.00}{##1}}}
\expandafter\def\csname PY@tok@kt\endcsname{\def\PY@tc##1{\textcolor[rgb]{0.69,0.00,0.25}{##1}}}
\expandafter\def\csname PY@tok@o\endcsname{\def\PY@tc##1{\textcolor[rgb]{0.40,0.40,0.40}{##1}}}
\expandafter\def\csname PY@tok@ow\endcsname{\let\PY@bf=\textbf\def\PY@tc##1{\textcolor[rgb]{0.67,0.13,1.00}{##1}}}
\expandafter\def\csname PY@tok@nb\endcsname{\def\PY@tc##1{\textcolor[rgb]{0.00,0.50,0.00}{##1}}}
\expandafter\def\csname PY@tok@nf\endcsname{\def\PY@tc##1{\textcolor[rgb]{0.00,0.00,1.00}{##1}}}
\expandafter\def\csname PY@tok@nc\endcsname{\let\PY@bf=\textbf\def\PY@tc##1{\textcolor[rgb]{0.00,0.00,1.00}{##1}}}
\expandafter\def\csname PY@tok@nn\endcsname{\let\PY@bf=\textbf\def\PY@tc##1{\textcolor[rgb]{0.00,0.00,1.00}{##1}}}
\expandafter\def\csname PY@tok@ne\endcsname{\let\PY@bf=\textbf\def\PY@tc##1{\textcolor[rgb]{0.82,0.25,0.23}{##1}}}
\expandafter\def\csname PY@tok@nv\endcsname{\def\PY@tc##1{\textcolor[rgb]{0.10,0.09,0.49}{##1}}}
\expandafter\def\csname PY@tok@no\endcsname{\def\PY@tc##1{\textcolor[rgb]{0.53,0.00,0.00}{##1}}}
\expandafter\def\csname PY@tok@nl\endcsname{\def\PY@tc##1{\textcolor[rgb]{0.63,0.63,0.00}{##1}}}
\expandafter\def\csname PY@tok@ni\endcsname{\let\PY@bf=\textbf\def\PY@tc##1{\textcolor[rgb]{0.60,0.60,0.60}{##1}}}
\expandafter\def\csname PY@tok@na\endcsname{\def\PY@tc##1{\textcolor[rgb]{0.49,0.56,0.16}{##1}}}
\expandafter\def\csname PY@tok@nt\endcsname{\let\PY@bf=\textbf\def\PY@tc##1{\textcolor[rgb]{0.00,0.50,0.00}{##1}}}
\expandafter\def\csname PY@tok@nd\endcsname{\def\PY@tc##1{\textcolor[rgb]{0.67,0.13,1.00}{##1}}}
\expandafter\def\csname PY@tok@s\endcsname{\def\PY@tc##1{\textcolor[rgb]{0.73,0.13,0.13}{##1}}}
\expandafter\def\csname PY@tok@sd\endcsname{\let\PY@it=\textit\def\PY@tc##1{\textcolor[rgb]{0.73,0.13,0.13}{##1}}}
\expandafter\def\csname PY@tok@si\endcsname{\let\PY@bf=\textbf\def\PY@tc##1{\textcolor[rgb]{0.73,0.40,0.53}{##1}}}
\expandafter\def\csname PY@tok@se\endcsname{\let\PY@bf=\textbf\def\PY@tc##1{\textcolor[rgb]{0.73,0.40,0.13}{##1}}}
\expandafter\def\csname PY@tok@sr\endcsname{\def\PY@tc##1{\textcolor[rgb]{0.73,0.40,0.53}{##1}}}
\expandafter\def\csname PY@tok@ss\endcsname{\def\PY@tc##1{\textcolor[rgb]{0.10,0.09,0.49}{##1}}}
\expandafter\def\csname PY@tok@sx\endcsname{\def\PY@tc##1{\textcolor[rgb]{0.00,0.50,0.00}{##1}}}
\expandafter\def\csname PY@tok@m\endcsname{\def\PY@tc##1{\textcolor[rgb]{0.40,0.40,0.40}{##1}}}
\expandafter\def\csname PY@tok@gh\endcsname{\let\PY@bf=\textbf\def\PY@tc##1{\textcolor[rgb]{0.00,0.00,0.50}{##1}}}
\expandafter\def\csname PY@tok@gu\endcsname{\let\PY@bf=\textbf\def\PY@tc##1{\textcolor[rgb]{0.50,0.00,0.50}{##1}}}
\expandafter\def\csname PY@tok@gd\endcsname{\def\PY@tc##1{\textcolor[rgb]{0.63,0.00,0.00}{##1}}}
\expandafter\def\csname PY@tok@gi\endcsname{\def\PY@tc##1{\textcolor[rgb]{0.00,0.63,0.00}{##1}}}
\expandafter\def\csname PY@tok@gr\endcsname{\def\PY@tc##1{\textcolor[rgb]{1.00,0.00,0.00}{##1}}}
\expandafter\def\csname PY@tok@ge\endcsname{\let\PY@it=\textit}
\expandafter\def\csname PY@tok@gs\endcsname{\let\PY@bf=\textbf}
\expandafter\def\csname PY@tok@gp\endcsname{\let\PY@bf=\textbf\def\PY@tc##1{\textcolor[rgb]{0.00,0.00,0.50}{##1}}}
\expandafter\def\csname PY@tok@go\endcsname{\def\PY@tc##1{\textcolor[rgb]{0.53,0.53,0.53}{##1}}}
\expandafter\def\csname PY@tok@gt\endcsname{\def\PY@tc##1{\textcolor[rgb]{0.00,0.27,0.87}{##1}}}
\expandafter\def\csname PY@tok@err\endcsname{\def\PY@bc##1{\setlength{\fboxsep}{0pt}\fcolorbox[rgb]{1.00,0.00,0.00}{1,1,1}{\strut ##1}}}
\expandafter\def\csname PY@tok@kc\endcsname{\let\PY@bf=\textbf\def\PY@tc##1{\textcolor[rgb]{0.00,0.50,0.00}{##1}}}
\expandafter\def\csname PY@tok@kd\endcsname{\let\PY@bf=\textbf\def\PY@tc##1{\textcolor[rgb]{0.00,0.50,0.00}{##1}}}
\expandafter\def\csname PY@tok@kn\endcsname{\let\PY@bf=\textbf\def\PY@tc##1{\textcolor[rgb]{0.00,0.50,0.00}{##1}}}
\expandafter\def\csname PY@tok@kr\endcsname{\let\PY@bf=\textbf\def\PY@tc##1{\textcolor[rgb]{0.00,0.50,0.00}{##1}}}
\expandafter\def\csname PY@tok@bp\endcsname{\def\PY@tc##1{\textcolor[rgb]{0.00,0.50,0.00}{##1}}}
\expandafter\def\csname PY@tok@fm\endcsname{\def\PY@tc##1{\textcolor[rgb]{0.00,0.00,1.00}{##1}}}
\expandafter\def\csname PY@tok@vc\endcsname{\def\PY@tc##1{\textcolor[rgb]{0.10,0.09,0.49}{##1}}}
\expandafter\def\csname PY@tok@vg\endcsname{\def\PY@tc##1{\textcolor[rgb]{0.10,0.09,0.49}{##1}}}
\expandafter\def\csname PY@tok@vi\endcsname{\def\PY@tc##1{\textcolor[rgb]{0.10,0.09,0.49}{##1}}}
\expandafter\def\csname PY@tok@vm\endcsname{\def\PY@tc##1{\textcolor[rgb]{0.10,0.09,0.49}{##1}}}
\expandafter\def\csname PY@tok@sa\endcsname{\def\PY@tc##1{\textcolor[rgb]{0.73,0.13,0.13}{##1}}}
\expandafter\def\csname PY@tok@sb\endcsname{\def\PY@tc##1{\textcolor[rgb]{0.73,0.13,0.13}{##1}}}
\expandafter\def\csname PY@tok@sc\endcsname{\def\PY@tc##1{\textcolor[rgb]{0.73,0.13,0.13}{##1}}}
\expandafter\def\csname PY@tok@dl\endcsname{\def\PY@tc##1{\textcolor[rgb]{0.73,0.13,0.13}{##1}}}
\expandafter\def\csname PY@tok@s2\endcsname{\def\PY@tc##1{\textcolor[rgb]{0.73,0.13,0.13}{##1}}}
\expandafter\def\csname PY@tok@sh\endcsname{\def\PY@tc##1{\textcolor[rgb]{0.73,0.13,0.13}{##1}}}
\expandafter\def\csname PY@tok@s1\endcsname{\def\PY@tc##1{\textcolor[rgb]{0.73,0.13,0.13}{##1}}}
\expandafter\def\csname PY@tok@mb\endcsname{\def\PY@tc##1{\textcolor[rgb]{0.40,0.40,0.40}{##1}}}
\expandafter\def\csname PY@tok@mf\endcsname{\def\PY@tc##1{\textcolor[rgb]{0.40,0.40,0.40}{##1}}}
\expandafter\def\csname PY@tok@mh\endcsname{\def\PY@tc##1{\textcolor[rgb]{0.40,0.40,0.40}{##1}}}
\expandafter\def\csname PY@tok@mi\endcsname{\def\PY@tc##1{\textcolor[rgb]{0.40,0.40,0.40}{##1}}}
\expandafter\def\csname PY@tok@il\endcsname{\def\PY@tc##1{\textcolor[rgb]{0.40,0.40,0.40}{##1}}}
\expandafter\def\csname PY@tok@mo\endcsname{\def\PY@tc##1{\textcolor[rgb]{0.40,0.40,0.40}{##1}}}
\expandafter\def\csname PY@tok@ch\endcsname{\let\PY@it=\textit\def\PY@tc##1{\textcolor[rgb]{0.25,0.50,0.50}{##1}}}
\expandafter\def\csname PY@tok@cm\endcsname{\let\PY@it=\textit\def\PY@tc##1{\textcolor[rgb]{0.25,0.50,0.50}{##1}}}
\expandafter\def\csname PY@tok@cpf\endcsname{\let\PY@it=\textit\def\PY@tc##1{\textcolor[rgb]{0.25,0.50,0.50}{##1}}}
\expandafter\def\csname PY@tok@c1\endcsname{\let\PY@it=\textit\def\PY@tc##1{\textcolor[rgb]{0.25,0.50,0.50}{##1}}}
\expandafter\def\csname PY@tok@cs\endcsname{\let\PY@it=\textit\def\PY@tc##1{\textcolor[rgb]{0.25,0.50,0.50}{##1}}}

\def\PYZbs{\char`\\}
\def\PYZus{\char`\_}
\def\PYZob{\char`\{}
\def\PYZcb{\char`\}}
\def\PYZca{\char`\^}
\def\PYZam{\char`\&}
\def\PYZlt{\char`\<}
\def\PYZgt{\char`\>}
\def\PYZsh{\char`\#}
\def\PYZpc{\char`\%}
\def\PYZdl{\char`\$}
\def\PYZhy{\char`\-}
\def\PYZsq{\char`\'}
\def\PYZdq{\char`\"}
\def\PYZti{\char`\~}
% for compatibility with earlier versions
\def\PYZat{@}
\def\PYZlb{[}
\def\PYZrb{]}
\makeatother


    % Exact colors from NB
    \definecolor{incolor}{rgb}{0.0, 0.0, 0.5}
    \definecolor{outcolor}{rgb}{0.545, 0.0, 0.0}



    
    % Prevent overflowing lines due to hard-to-break entities
    \sloppy 
    % Setup hyperref package
    \hypersetup{
      breaklinks=true,  % so long urls are correctly broken across lines
      colorlinks=true,
      urlcolor=urlcolor,
      linkcolor=linkcolor,
      citecolor=citecolor,
      }
    % Slightly bigger margins than the latex defaults
    
    \geometry{verbose,tmargin=1in,bmargin=1in,lmargin=1in,rmargin=1in}
    
    

    \begin{document}
    
    
    \maketitle
    
    

    
    \hypertarget{exam-1}{%
\section{Exam 1}\label{exam-1}}

\hypertarget{read-each-question-carefully-and-use-r-to-show-how-you-calculated-each-answer}{%
\subsubsection{Read each question carefully and use R to show how you
calculated each
answer}\label{read-each-question-carefully-and-use-r-to-show-how-you-calculated-each-answer}}

    \hypertarget{in-a-congested-city-when-it-rains-which-happens-one-third-of-the-days-there-is-50-probability-that-there-will-be-heavy-traffic.-on-the-other-hand-if-it-doesnt-rain-then-the-probability-gets-reduced-to-only-25.-now-if-its-rainy-and-there-is-heavy-traffic-there-is-50-chance-hat-i-will-arrive-late-to-work-but-only-18-if-it-is-sunny-and-no-traffic.-i-will-be-late-only-14-of-the-time-if-there-is-rain-and-no-traffic-or-not-rain-and-traffic.}{%
\subsubsection{1. In a congested city when it rains (which happens one
third of the days), there is 50\% probability that there will be heavy
traffic. On the other hand, if it doesn't rain, then the probability
gets reduced to only 25\%. Now, if its rainy and there is heavy traffic,
there is 50\% chance hat I will arrive late to work, but only 1/8 if it
is sunny and no traffic. I will be late only 1/4 of the time if there is
rain and no traffic or not rain and
traffic.}\label{in-a-congested-city-when-it-rains-which-happens-one-third-of-the-days-there-is-50-probability-that-there-will-be-heavy-traffic.-on-the-other-hand-if-it-doesnt-rain-then-the-probability-gets-reduced-to-only-25.-now-if-its-rainy-and-there-is-heavy-traffic-there-is-50-chance-hat-i-will-arrive-late-to-work-but-only-18-if-it-is-sunny-and-no-traffic.-i-will-be-late-only-14-of-the-time-if-there-is-rain-and-no-traffic-or-not-rain-and-traffic.}}

 If I today I arrived late to work, what is the probability that we had
rain that day.?

Hint (you can use tree diagrams and conditional probabilities to find
the answer)

    \begin{Verbatim}[commandchars=\\\{\}]
{\color{incolor}In [{\color{incolor}3}]:} \PY{c+c1}{\PYZsh{}install.packages(\PYZdq{}igraph\PYZdq{})}
        \PY{k+kn}{library}\PY{p}{(}igraph\PY{p}{)}
\end{Verbatim}


    \begin{Verbatim}[commandchars=\\\{\}]
{\color{incolor}In [{\color{incolor}4}]:} g \PY{o}{\PYZlt{}\PYZhy{}} graph.tree\PY{p}{(}n \PY{o}{=} \PY{l+m}{2}\PY{o}{\PYZca{}}\PY{l+m}{4} \PY{o}{\PYZhy{}} \PY{l+m}{1}\PY{p}{,} children \PY{o}{=} \PY{l+m}{2}\PY{p}{)} 
        \PY{c+c1}{\PYZsh{} we need four levels including the root (15 nodes), and each parent having two children}
        \PY{c+c1}{\PYZsh{} Rain/Not Rain; Heavy Traffic/ Not Heavy Traffic; Late/Not Late}
        
        \PY{c+c1}{\PYZsh{}\PYZsh{}Lets add the node labels}
        n\PYZus{}l \PY{o}{=} \PY{k+kt}{c}\PY{p}{(}\PY{l+s}{\PYZdq{}}\PY{l+s}{\PYZdq{}}\PY{p}{,}\PY{l+s}{\PYZdq{}}\PY{l+s}{Rain\PYZdq{}}\PY{p}{,}\PY{l+s}{\PYZdq{}}\PY{l+s}{Sunny\PYZdq{}}\PY{p}{,}\PY{l+s}{\PYZdq{}}\PY{l+s}{Traffic\PYZdq{}}\PY{p}{,}\PY{l+s}{\PYZdq{}}\PY{l+s}{Not Traffic\PYZdq{}}\PY{p}{,}\PY{l+s}{\PYZdq{}}\PY{l+s}{Traffic\PYZdq{}}\PY{p}{,}\PY{l+s}{\PYZdq{}}\PY{l+s}{Not Traffic\PYZdq{}}\PY{p}{)}
        node\PYZus{}labels \PY{o}{\PYZlt{}\PYZhy{}} \PY{k+kt}{c}\PY{p}{(}n\PYZus{}l\PY{p}{,}\PY{k+kp}{replicate}\PY{p}{(}\PY{l+m}{4}\PY{p}{,}\PY{k+kt}{c}\PY{p}{(}\PY{l+s}{\PYZdq{}}\PY{l+s}{Late\PYZdq{}}\PY{p}{,}\PY{l+s}{\PYZdq{}}\PY{l+s}{Not Late\PYZdq{}}\PY{p}{)}\PY{p}{)}\PY{p}{)}
        node\PYZus{}labels
\end{Verbatim}


    \begin{enumerate*}
\item ''
\item 'Rain'
\item 'Sunny'
\item 'Traffic'
\item 'Not Traffic'
\item 'Traffic'
\item 'Not Traffic'
\item 'Late'
\item 'Not Late'
\item 'Late'
\item 'Not Late'
\item 'Late'
\item 'Not Late'
\item 'Late'
\item 'Not Late'
\end{enumerate*}


    
    \begin{Verbatim}[commandchars=\\\{\}]
{\color{incolor}In [{\color{incolor}5}]:} edge\PYZus{}labels \PY{o}{\PYZlt{}\PYZhy{}} \PY{k+kt}{c}\PY{p}{(}\PY{l+s}{\PYZdq{}}\PY{l+s}{1/3\PYZdq{}}\PY{p}{,}\PY{l+s}{\PYZdq{}}\PY{l+s}{2/3\PYZdq{}}\PY{p}{,}\PY{l+s}{\PYZdq{}}\PY{l+s}{1/2\PYZdq{}}\PY{p}{,}\PY{l+s}{\PYZdq{}}\PY{l+s}{1/2\PYZdq{}}\PY{p}{,}\PY{l+s}{\PYZdq{}}\PY{l+s}{1/4\PYZdq{}}\PY{p}{,}\PY{l+s}{\PYZdq{}}\PY{l+s}{3/4\PYZdq{}}\PY{p}{,}\PY{l+s}{\PYZdq{}}\PY{l+s}{1/2\PYZdq{}}\PY{p}{,}\PY{l+s}{\PYZdq{}}\PY{l+s}{1/2\PYZdq{}}\PY{p}{,}\PY{l+s}{\PYZdq{}}\PY{l+s}{1/4\PYZdq{}}\PY{p}{,}\PY{l+s}{\PYZdq{}}\PY{l+s}{3/4\PYZdq{}}\PY{p}{,}\PY{l+s}{\PYZdq{}}\PY{l+s}{1/4\PYZdq{}}\PY{p}{,}\PY{l+s}{\PYZdq{}}\PY{l+s}{3/4\PYZdq{}}\PY{p}{,}\PY{l+s}{\PYZdq{}}\PY{l+s}{1/8\PYZdq{}}\PY{p}{,}\PY{l+s}{\PYZdq{}}\PY{l+s}{7/8\PYZdq{}}\PY{p}{)}
        edge\PYZus{}label2 \PY{o}{=} edge\PYZus{}labels
\end{Verbatim}


    \begin{Verbatim}[commandchars=\\\{\}]
{\color{incolor}In [{\color{incolor}6}]:} \PY{c+c1}{\PYZsh{}Assign Color}
        V\PY{p}{(}g\PY{p}{)}\PY{o}{\PYZdl{}}color \PY{o}{\PYZlt{}\PYZhy{}} \PY{l+s}{\PYZdq{}}\PY{l+s}{\PYZsh{}C4D8E2\PYZdq{}}
        \PY{c+c1}{\PYZsh{}V(g)\PYZdl{}color[3] \PYZlt{}\PYZhy{} \PYZdq{}white\PYZdq{}}
        \PY{c+c1}{\PYZsh{}V(g)\PYZdl{}color[4] \PYZlt{}\PYZhy{} \PYZdq{}green\PYZdq{}}
        
        \PY{c+c1}{\PYZsh{}assign position}
        coords \PY{o}{\PYZlt{}\PYZhy{}} layout\PYZus{}\PY{p}{(}g\PY{p}{,} as\PYZus{}tree\PY{p}{(}\PY{p}{)}\PY{p}{)}
        coord2 \PY{o}{=} \PY{k+kt}{matrix}\PY{p}{(}\PY{k+kt}{c}\PY{p}{(}\PY{o}{\PYZhy{}}coords\PY{p}{[}\PY{p}{,}\PY{l+m}{2}\PY{p}{]}\PY{p}{,}\PY{o}{\PYZhy{}}coords\PY{p}{[}\PY{p}{,}\PY{l+m}{1}\PY{p}{]}\PY{p}{)}\PY{p}{,}ncol \PY{o}{=} \PY{l+m}{2}\PY{p}{)}
\end{Verbatim}


    \begin{Verbatim}[commandchars=\\\{\}]
{\color{incolor}In [{\color{incolor}7}]:} plot\PY{p}{(}g\PY{p}{,}
             layout \PY{o}{=} coord2\PY{p}{,}           \PY{c+c1}{\PYZsh{} draw graph as tree}
             vertex.size \PY{o}{=} \PY{l+m}{20}\PY{p}{,}                  \PY{c+c1}{\PYZsh{} node size}
             vertex.color \PY{o}{=} V\PY{p}{(}g\PY{p}{)}\PY{o}{\PYZdl{}}color\PY{p}{,}          \PY{c+c1}{\PYZsh{} node color}
             vertex.label \PY{o}{=} node\PYZus{}labels\PY{p}{,}        \PY{c+c1}{\PYZsh{} node labels}
             vertex.label.cex \PY{o}{=} \PY{l+m}{1}\PY{p}{,}             \PY{c+c1}{\PYZsh{} node label size}
             vertex.label.family \PY{o}{=} \PY{l+s}{\PYZdq{}}\PY{l+s}{Helvetica\PYZdq{}}\PY{p}{,} \PY{c+c1}{\PYZsh{} node label family}
             vertex.label.font \PY{o}{=} \PY{l+m}{2}\PY{p}{,}             \PY{c+c1}{\PYZsh{} node label type (bold)}
             vertex.label.color \PY{o}{=} \PY{l+s}{\PYZsq{}}\PY{l+s}{\PYZsh{}000000\PYZsq{}}\PY{p}{,}    \PY{c+c1}{\PYZsh{} node label size}
             edge.label \PY{o}{=} edge\PYZus{}label2\PY{p}{,}          \PY{c+c1}{\PYZsh{} edge labels}
             edge.label.cex \PY{o}{=} \PY{l+m}{.7}\PY{p}{,}               \PY{c+c1}{\PYZsh{} edge label size}
             edge.label.family \PY{o}{=} \PY{l+s}{\PYZdq{}}\PY{l+s}{Helvetica\PYZdq{}}\PY{p}{,}   \PY{c+c1}{\PYZsh{} edge label family}
             edge.label.font \PY{o}{=} \PY{l+m}{1}\PY{p}{,}               \PY{c+c1}{\PYZsh{} edge label font type (bold)}
             edge.label.color \PY{o}{=} \PY{l+s}{\PYZsq{}}\PY{l+s}{\PYZsh{}000000\PYZsq{}}\PY{p}{,}      \PY{c+c1}{\PYZsh{} edge label color}
             edge.arrow.size \PY{o}{=} \PY{l+m}{0.2}\PY{p}{,}              \PY{c+c1}{\PYZsh{} arrow size}
             edge.arrow.width \PY{o}{=} \PY{l+m}{1}              \PY{c+c1}{\PYZsh{} arrow width}
        \PY{p}{)}
\end{Verbatim}


    \begin{center}
    \adjustimage{max size={0.9\linewidth}{0.9\paperheight}}{output_6_0.png}
    \end{center}
    { \hspace*{\fill} \\}
    
    \begin{Verbatim}[commandchars=\\\{\}]
{\color{incolor}In [{\color{incolor}8}]:} P\PY{p}{(}L\PY{o}{|}R\PY{p}{)}\PY{o}{*}P\PY{p}{(}R\PY{p}{)}\PY{o}{=}\PY{p}{(}\PY{l+m}{1}\PY{o}{/}\PY{l+m}{4}\PY{l+m}{+1}\PY{o}{/}\PY{l+m}{2}\PY{o}{*}\PY{l+m}{1}\PY{o}{/}\PY{l+m}{4}\PY{p}{)}\PY{o}{*}\PY{l+m}{1}\PY{o}{/}\PY{l+m}{3}\PY{o}{=}\PY{l+m}{1}\PY{o}{/}\PY{l+m}{12}\PY{l+m}{+1}\PY{o}{/}\PY{l+m}{24}\PY{o}{=}\PY{l+m}{1}\PY{o}{/}\PY{l+m}{8}
        P\PY{p}{(}Late\PY{p}{)}\PY{o}{=}\PY{l+m}{1}\PY{o}{/}\PY{l+m}{3}\PY{o}{*}\PY{l+m}{1}\PY{o}{/}\PY{l+m}{4}\PY{l+m}{+1}\PY{o}{/}\PY{l+m}{24}\PY{l+m}{+2}\PY{o}{/}\PY{l+m}{3}\PY{o}{*}\PY{l+m}{1}\PY{o}{/}\PY{l+m}{16}\PY{l+m}{+2}\PY{o}{/}\PY{l+m}{3}\PY{o}{*}\PY{l+m}{3}\PY{o}{/}\PY{l+m}{4}\PY{o}{*}\PY{l+m}{1}\PY{o}{/}\PY{l+m}{8}\PY{o}{=}\PY{l+m}{1}\PY{o}{/}\PY{l+m}{12}\PY{l+m}{+1}\PY{o}{/}\PY{l+m}{24}\PY{l+m}{+1}\PY{o}{/}\PY{l+m}{24}\PY{l+m}{+1}\PY{o}{/}\PY{l+m}{16}\PY{o}{=}\PY{l+m}{11}\PY{o}{/}\PY{l+m}{48}
        P\PY{p}{(}Rain\PY{o}{|}Late\PY{p}{)}\PY{o}{=}P\PY{p}{(}L\PY{o}{|}R\PY{p}{)}\PY{o}{*}P\PY{p}{(}R\PY{p}{)}\PY{o}{/}P\PY{p}{(}Late\PY{p}{)}\PY{o}{=} \PY{l+m}{6}\PY{o}{/}\PY{l+m}{11}
\end{Verbatim}


    \begin{Verbatim}[commandchars=\\\{\}]

        Error in 1/12 + 1/24 = 1/8: target of assignment expands to non-language object
    Traceback:


    \end{Verbatim}

    \hypertarget{we-classify-2000-email-in-two-groups-1000-emails-as-spam-and-1000-emails-as-non-spam.-210-of-the-spam-emails-contained-the-phrase-this-isnt-spam-99-had-the-word-prize-and-110-the-word-prince.-of-the-99-that-contained-the-word-prize-79-also-contained-the-word-prince.-on-the-other-hand-of-the-1000-non-spam-emails-only-23-had-the-phrase-this-isnt-spam-80-the-word-prize-and-110-the-word-prince.-of-the-80-that-contained-the-word-prize-8-also-contained-the-word-prince.}{%
\subsubsection{2. we classify 2000 email in two groups: 1000 emails as
spam and 1000 emails as non-spam. 210 of the spam emails contained the
phrase This isn't spam, 99 had the word prize and 110 the word prince.
Of the 99 that contained the word prize, 79 also contained the word
prince. On the other hand, of the 1000 non-spam emails, only 23 had the
phrase this isn't spam, 80 the word prize and 110 the word prince. Of
the 80 that contained the word prize 8 also contained the word
prince.}\label{we-classify-2000-email-in-two-groups-1000-emails-as-spam-and-1000-emails-as-non-spam.-210-of-the-spam-emails-contained-the-phrase-this-isnt-spam-99-had-the-word-prize-and-110-the-word-prince.-of-the-99-that-contained-the-word-prize-79-also-contained-the-word-prince.-on-the-other-hand-of-the-1000-non-spam-emails-only-23-had-the-phrase-this-isnt-spam-80-the-word-prize-and-110-the-word-prince.-of-the-80-that-contained-the-word-prize-8-also-contained-the-word-prince.}}

 Assuming that the a priori probability of any message being spam is
0.5, what is the probability that an email is spam given it contains the
phrase This isn't spam

    \begin{Verbatim}[commandchars=\\\{\}]
{\color{incolor}In [{\color{incolor} }]:} A\PY{o}{:} Contains This is not spam
        P\PY{p}{(}A\PY{o}{|}spam\PY{p}{)}\PY{o}{=}\PY{l+m}{210}\PY{o}{/}\PY{l+m}{1000}\PY{o}{=}\PY{l+m}{0.21}
        P\PY{p}{(}A\PY{p}{)}\PY{o}{=}\PY{p}{(}\PY{l+m}{210}\PY{l+m}{+23}\PY{p}{)}\PY{o}{/}\PY{l+m}{2000}\PY{o}{=}\PY{l+m}{0.1165}
        P\PY{p}{(}spam\PY{o}{|}A\PY{p}{)}\PY{o}{=}P\PY{p}{(}A\PY{o}{|}spam\PY{p}{)}\PY{o}{*}P\PY{p}{(}spam\PY{p}{)}\PY{o}{/}P\PY{p}{(}A\PY{p}{)}\PY{o}{=}\PY{l+m}{0.21}\PY{o}{*}\PY{l+m}{0.5}\PY{o}{/}\PY{l+m}{0.1165}\PY{o}{=}\PY{l+m}{0.90}
\end{Verbatim}


    \hypertarget{the-blood-transfusion-service-center-in-hsin-chu-city-taiwan-collects-data-to-understand-donation-habits-from-a-center-that-passes-their-blood-transfusion-service-bus-to-one-university-in-hsin-chu-city.-data-is-collected-on-whether-the-person-donates-or-not-in-march-as-a-binary-variable-and-multiple-categorical-variables-data-obtained-from-httparchive.ics.uci.edumlmachine-learning-databasesblood-transfusion}{%
\subsubsection{\texorpdfstring{3. The Blood Transfusion Service Center
in Hsin-Chu City, Taiwan collects data to understand donation habits
from a center that passes their blood transfusion service bus to one
university in Hsin-Chu City. Data is collected on whether the person
donates or not in March as a binary variable, and multiple categorical
variables (data obtained from
\url{http://archive.ics.uci.edu/ml/machine-learning-databases/blood-transfusion/})}{3. The Blood Transfusion Service Center in Hsin-Chu City, Taiwan collects data to understand donation habits from a center that passes their blood transfusion service bus to one university in Hsin-Chu City. Data is collected on whether the person donates or not in March as a binary variable, and multiple categorical variables (data obtained from http://archive.ics.uci.edu/ml/machine-learning-databases/blood-transfusion/)}}\label{the-blood-transfusion-service-center-in-hsin-chu-city-taiwan-collects-data-to-understand-donation-habits-from-a-center-that-passes-their-blood-transfusion-service-bus-to-one-university-in-hsin-chu-city.-data-is-collected-on-whether-the-person-donates-or-not-in-march-as-a-binary-variable-and-multiple-categorical-variables-data-obtained-from-httparchive.ics.uci.edumlmachine-learning-databasesblood-transfusion}}

\hypertarget{r-recency---months-since-last-donation}{%
\paragraph{R (Recency - months since last
donation),}\label{r-recency---months-since-last-donation}}

\hypertarget{f-frequency---total-number-of-donation}{%
\paragraph{F (Frequency - total number of
donation),}\label{f-frequency---total-number-of-donation}}

\hypertarget{m-monetary---total-blood-donated-in-c.c.}{%
\paragraph{M (Monetary - total blood donated in
c.c.),}\label{m-monetary---total-blood-donated-in-c.c.}}

\hypertarget{t-time---months-since-first-donation-and}{%
\paragraph{T (Time - months since first donation),
and}\label{t-time---months-since-first-donation-and}}

 Using contingency tables, calculate the probability that a person
donates blood in march given that they donated blood in a Frequency
between 18 and 33 times The frequency variable should be converted to a
three way categorical variable 1 = 1-17; 2 = 18-33; 3 = 34-50

    \begin{Verbatim}[commandchars=\\\{\}]
{\color{incolor}In [{\color{incolor}50}]:} transfusion \PY{o}{=} read.csv\PY{p}{(}file \PY{o}{=} \PY{l+s}{\PYZdq{}}\PY{l+s}{transfusion.csv\PYZdq{}}\PY{p}{,}header\PY{o}{=}\PY{k+kc}{TRUE}\PY{p}{,} sep\PY{o}{=}\PY{l+s}{\PYZdq{}}\PY{l+s}{,\PYZdq{}}\PY{p}{)}
         \PY{c+c1}{\PYZsh{}transfusion}
\end{Verbatim}


    \begin{Verbatim}[commandchars=\\\{\}]
{\color{incolor}In [{\color{incolor}51}]:} transfusion\PY{o}{\PYZdl{}}Frequency\PY{o}{\PYZlt{}\PYZhy{}}\PY{k+kp}{cut}\PY{p}{(}transfusion\PY{o}{\PYZdl{}}Frequency\PY{p}{,} \PY{k+kt}{c}\PY{p}{(}\PY{l+m}{0}\PY{p}{,}\PY{l+m}{17}\PY{p}{,}\PY{l+m}{33}\PY{p}{,}\PY{l+m}{50}\PY{p}{)}\PY{p}{,}labels\PY{o}{=}\PY{k+kt}{c}\PY{p}{(}\PY{l+m}{1}\PY{o}{:}\PY{l+m}{3}\PY{p}{)}\PY{p}{)}
         \PY{c+c1}{\PYZsh{}transfusion}
\end{Verbatim}


    \begin{Verbatim}[commandchars=\\\{\}]
{\color{incolor}In [{\color{incolor}22}]:} install.packages\PY{p}{(}\PY{l+s}{\PYZdq{}}\PY{l+s}{dplyr\PYZdq{}}\PY{p}{)}
\end{Verbatim}


    \begin{Verbatim}[commandchars=\\\{\}]

The downloaded binary packages are in
	/var/folders/79/jxb90vv11gvb4bfw9cs5\_kg00000gn/T//RtmpUSyLLQ/downloaded\_packages

    \end{Verbatim}

    \begin{Verbatim}[commandchars=\\\{\}]
{\color{incolor}In [{\color{incolor}24}]:} install.packages\PY{p}{(}\PY{l+s}{\PYZdq{}}\PY{l+s}{igraph\PYZdq{}}\PY{p}{)}
\end{Verbatim}


    \begin{Verbatim}[commandchars=\\\{\}]

The downloaded binary packages are in
	/var/folders/79/jxb90vv11gvb4bfw9cs5\_kg00000gn/T//RtmpUSyLLQ/downloaded\_packages

    \end{Verbatim}

    \begin{Verbatim}[commandchars=\\\{\}]
Warning message:
“package ‘stats’ is not available (for R version 3.4.3)”Warning message:
“package ‘stats’ is a base package, and should not be updated”Warning message:
“package ‘base’ is not available (for R version 3.4.3)”Warning message:
“package ‘base’ is a base package, and should not be updated”
    \end{Verbatim}

    \begin{Verbatim}[commandchars=\\\{\}]
{\color{incolor}In [{\color{incolor}25}]:} \PY{k+kn}{library}\PY{p}{(}\PY{l+s}{\PYZdq{}}\PY{l+s}{ggplot2\PYZdq{}}\PY{p}{)}
         \PY{k+kn}{library}\PY{p}{(}\PY{l+s}{\PYZdq{}}\PY{l+s}{dplyr\PYZdq{}}\PY{p}{)}
         \PY{k+kn}{library}\PY{p}{(}\PY{l+s}{\PYZdq{}}\PY{l+s}{reshape2\PYZdq{}}\PY{p}{)}
         \PY{k+kn}{library}\PY{p}{(}\PY{l+s}{\PYZdq{}}\PY{l+s}{knitr\PYZdq{}}\PY{p}{)}
\end{Verbatim}


    \begin{Verbatim}[commandchars=\\\{\}]
{\color{incolor}In [{\color{incolor}32}]:} transfusion.March.freq.df \PY{o}{\PYZlt{}\PYZhy{}}
           transfusion \PY{o}{\PYZpc{}\PYZgt{}\PYZpc{}}
           group\PYZus{}by\PY{p}{(}Donated\PYZus{}In\PYZus{}March\PY{p}{,} Frequency\PY{p}{)} \PY{o}{\PYZpc{}\PYZgt{}\PYZpc{}}
           summarize\PY{p}{(}n \PY{o}{=} n\PY{p}{(}\PY{p}{)}\PY{p}{)}
\end{Verbatim}


    \begin{Verbatim}[commandchars=\\\{\}]
{\color{incolor}In [{\color{incolor}33}]:} transfusion.March.freq.df \PY{o}{\PYZpc{}\PYZgt{}\PYZpc{}}
           dcast\PY{p}{(}Donated\PYZus{}In\PYZus{}March \PY{o}{\PYZti{}} Frequency\PY{p}{,} value.var \PY{o}{=} \PY{l+s}{\PYZdq{}}\PY{l+s}{n\PYZdq{}}\PY{p}{)} \PY{o}{\PYZpc{}\PYZgt{}\PYZpc{}}
           kable\PY{p}{(}align \PY{o}{=} \PY{l+s}{\PYZdq{}}\PY{l+s}{l\PYZdq{}}\PY{p}{,} format \PY{o}{=} \PY{l+s}{\PYZdq{}}\PY{l+s}{markdown\PYZdq{}}\PY{p}{,}
                 table.attr\PY{o}{=}\PY{l+s}{\PYZsq{}}\PY{l+s}{class=\PYZdq{}table table\PYZhy{}striped table\PYZhy{}hover\PYZdq{}\PYZsq{}}\PY{p}{)}
\end{Verbatim}


    
    \begin{verbatim}


|Donated_In_March |1   |2  |3  |
|:----------------|:---|:--|:--|
|0                |561 |7  |2  |
|1                |165 |8  |5  |
    \end{verbatim}

    
    \begin{Verbatim}[commandchars=\\\{\}]
{\color{incolor}In [{\color{incolor}35}]:} transfusion.March.freq.prop.df \PY{o}{\PYZlt{}\PYZhy{}} 
           transfusion.March.freq.df \PY{o}{\PYZpc{}\PYZgt{}\PYZpc{}}
           ungroup\PY{p}{(}\PY{p}{)} \PY{o}{\PYZpc{}\PYZgt{}\PYZpc{}}
           mutate\PY{p}{(}prop \PY{o}{=} n \PY{o}{/} \PY{k+kp}{sum}\PY{p}{(}n\PY{p}{)}\PY{p}{)}
         
         transfusion.March.freq.prop.df \PY{o}{\PYZpc{}\PYZgt{}\PYZpc{}}
           dcast\PY{p}{(}Donated\PYZus{}In\PYZus{}March \PY{o}{\PYZti{}} Frequency\PY{p}{,} value.var \PY{o}{=} \PY{l+s}{\PYZdq{}}\PY{l+s}{prop\PYZdq{}}\PY{p}{)} \PY{o}{\PYZpc{}\PYZgt{}\PYZpc{}}
           kable\PY{p}{(}align \PY{o}{=} \PY{l+s}{\PYZdq{}}\PY{l+s}{l\PYZdq{}}\PY{p}{,} format \PY{o}{=} \PY{l+s}{\PYZdq{}}\PY{l+s}{markdown\PYZdq{}}\PY{p}{,} 
                 table.attr \PY{o}{=} \PY{l+s}{\PYZsq{}}\PY{l+s}{class=\PYZdq{}table table\PYZhy{}striped table\PYZhy{}hover\PYZdq{}\PYZsq{}}\PY{p}{)}
\end{Verbatim}


    
    \begin{verbatim}


|Donated_In_March |1         |2         |3         |
|:----------------|:---------|:---------|:---------|
|0                |0.7500000 |0.0093583 |0.0026738 |
|1                |0.2205882 |0.0106952 |0.0066845 |
    \end{verbatim}

    
    \begin{Verbatim}[commandchars=\\\{\}]
{\color{incolor}In [{\color{incolor}48}]:} March.marginal.df \PY{o}{\PYZlt{}\PYZhy{}} 
           transfusion.March.freq.prop.df \PY{o}{\PYZpc{}\PYZgt{}\PYZpc{}}
           group\PYZus{}by\PY{p}{(}Donated\PYZus{}In\PYZus{}March\PY{p}{)} \PY{o}{\PYZpc{}\PYZgt{}\PYZpc{}}
           summarize\PY{p}{(}marginal \PY{o}{=} \PY{k+kp}{sum}\PY{p}{(}prop\PY{p}{)}\PY{p}{)}
         
         freq.marginal.df \PY{o}{\PYZlt{}\PYZhy{}} 
           transfusion.March.freq.prop.df \PY{o}{\PYZpc{}\PYZgt{}\PYZpc{}}
           group\PYZus{}by\PY{p}{(}Frequency\PY{p}{)} \PY{o}{\PYZpc{}\PYZgt{}\PYZpc{}}
           summarize\PY{p}{(}marginal \PY{o}{=} \PY{k+kp}{sum}\PY{p}{(}prop\PY{p}{)}\PY{p}{)}
         
         transfusion.March.freq.prop.df \PY{o}{\PYZpc{}\PYZgt{}\PYZpc{}}
           dcast\PY{p}{(}Donated\PYZus{}In\PYZus{}March \PY{o}{\PYZti{}} Frequency\PY{p}{,} value.var \PY{o}{=} \PY{l+s}{\PYZdq{}}\PY{l+s}{prop\PYZdq{}}\PY{p}{)} \PY{o}{\PYZpc{}\PYZgt{}\PYZpc{}}
           left\PYZus{}join\PY{p}{(}March.marginal.df\PY{p}{,} by \PY{o}{=} \PY{l+s}{\PYZdq{}}\PY{l+s}{Donated\PYZus{}In\PYZus{}March\PYZdq{}}\PY{p}{)} \PY{o}{\PYZpc{}\PYZgt{}\PYZpc{}}
         \PY{c+c1}{\PYZsh{}   bind\PYZus{}rows(}
         \PY{c+c1}{\PYZsh{}     freq.marginal.df \PYZpc{}\PYZgt{}\PYZpc{}}
         \PY{c+c1}{\PYZsh{}       mutate(Donated\PYZus{}In\PYZus{}March = \PYZdq{}marginal\PYZdq{}) \PYZpc{}\PYZgt{}\PYZpc{}}
         \PY{c+c1}{\PYZsh{}       dcast(Donated\PYZus{}In\PYZus{}March \PYZti{} Frequency, value.var = \PYZdq{}marginal\PYZdq{})}
         \PY{c+c1}{\PYZsh{}   ) \PYZpc{}\PYZgt{}\PYZpc{}}
           kable\PY{p}{(}align \PY{o}{=} \PY{l+s}{\PYZdq{}}\PY{l+s}{l\PYZdq{}}\PY{p}{,} format \PY{o}{=} \PY{l+s}{\PYZdq{}}\PY{l+s}{markdown\PYZdq{}}\PY{p}{,}
                 table.attr \PY{o}{=} \PY{l+s}{\PYZsq{}}\PY{l+s}{class=\PYZdq{}table table\PYZhy{}striped table\PYZhy{}hover\PYZdq{}\PYZsq{}}\PY{p}{)}
\end{Verbatim}


    
    \begin{verbatim}


|Donated_In_March |1         |2         |3         |marginal  |
|:----------------|:---------|:---------|:---------|:---------|
|0                |0.7500000 |0.0093583 |0.0026738 |0.7620321 |
|1                |0.2205882 |0.0106952 |0.0066845 |0.2379679 |
    \end{verbatim}

    
    \begin{Verbatim}[commandchars=\\\{\}]
{\color{incolor}In [{\color{incolor}44}]:} freq.marginal.df \PY{o}{\PYZpc{}\PYZgt{}\PYZpc{}}
               mutate\PY{p}{(}Donated\PYZus{}In\PYZus{}March \PY{o}{=} \PY{l+s}{\PYZdq{}}\PY{l+s}{marginal\PYZdq{}}\PY{p}{)} \PY{o}{\PYZpc{}\PYZgt{}\PYZpc{}}
               dcast\PY{p}{(}Donated\PYZus{}In\PYZus{}March \PY{o}{\PYZti{}} Frequency\PY{p}{,} value.var \PY{o}{=} \PY{l+s}{\PYZdq{}}\PY{l+s}{marginal\PYZdq{}}\PY{p}{)}
\end{Verbatim}


    \begin{tabular}{r|llll}
 Donated\_In\_March & 1 & 2 & 3\\
\hline
	 marginal    & 0.9705882   & 0.02005348  & 0.009358289\\
\end{tabular}


    
    2= Frequency between 18 and 33 times From this contingency table,
P(March and 2)=0.0106952, P(2)=0.02005348 P(March\textbar{}2) = P(March
and 2)/P(2) =0.0106952/0.02005348 = 0.53

    \hypertarget{in-a-class-there-are-18-math-majors-and-25-physics-majors.-12-math-majors-are-females-as-well-as-20-physics-majors}{%
\subsubsection{4. In a class there are 18 math majors and 25 physics
majors. 12 math majors are females as well as 20 physics
majors,}\label{in-a-class-there-are-18-math-majors-and-25-physics-majors.-12-math-majors-are-females-as-well-as-20-physics-majors}}

\hypertarget{find-the-probability-that-the-student-selected-at-random-is-a-math-major-or-a-male.}{%
\paragraph{Find the probability that the student selected at random is a
math major or a male.
}\label{find-the-probability-that-the-student-selected-at-random-is-a-math-major-or-a-male.}}

    \begin{Verbatim}[commandchars=\\\{\}]
{\color{incolor}In [{\color{incolor} }]:} P\PY{p}{(}math or male\PY{p}{)}\PY{o}{=}P\PY{p}{(}math\PY{p}{)}\PY{o}{+}P\PY{p}{(}male\PY{p}{)}\PY{o}{=}\PY{l+m}{18}\PY{o}{/}\PY{p}{(}\PY{l+m}{18}\PY{l+m}{+25}\PY{p}{)}\PY{o}{+}\PY{p}{(}\PY{p}{(}\PY{l+m}{18}\PY{l+m}{+25}\PY{p}{)}\PY{o}{\PYZhy{}}\PY{p}{(}\PY{l+m}{12}\PY{l+m}{+20}\PY{p}{)}\PY{p}{)}\PY{o}{/}\PY{p}{(}\PY{l+m}{18}\PY{l+m}{+25}\PY{p}{)}\PY{o}{=}\PY{l+m}{0.674}
\end{Verbatim}


    \hypertarget{there-are-6-cars-in-a-car-shop-out-which-3-are-defective.-if-2-cars-are-picked-randomly}{%
\subsubsection{5. There are 6 cars in a car shop out which 3 are
defective. If 2 cars are picked
randomly,}\label{there-are-6-cars-in-a-car-shop-out-which-3-are-defective.-if-2-cars-are-picked-randomly}}

\hypertarget{find-the-probability-that-at-least-one-is-defective.}{%
\paragraph{Find the probability that at least one is
defective.}\label{find-the-probability-that-at-least-one-is-defective.}}

    \begin{Verbatim}[commandchars=\\\{\}]
{\color{incolor}In [{\color{incolor} }]:} P \PY{o}{=} \PY{p}{(}\PY{l+m}{3}C2\PY{p}{)}\PY{o}{/}\PY{p}{(}\PY{l+m}{6}C2\PY{p}{)}\PY{o}{=}\PY{l+m}{3}\PY{o}{/}\PY{l+m}{15}\PY{o}{=}\PY{l+m}{0.2}
\end{Verbatim}


    \hypertarget{in-the-past-for-every-attempt-to-make-a-call-there-was-a-70-probability-of-getting-the-call.}{%
\subsubsection{6. In the past, for every attempt to make a call there
was a 70\% probability of getting the
call.}\label{in-the-past-for-every-attempt-to-make-a-call-there-was-a-70-probability-of-getting-the-call.}}

\hypertarget{a.-calculate-the-probability-of-having-12-successes-in-20-attempts.-b.-plot-the-distribution-and-describe-the-shape}{%
\paragraph{\texorpdfstring{ a. Calculate the probability of having 12
successes in 20 attempts. b. Plot the distribution and describe the
shape}{  a. Calculate the probability of having 12 successes in 20 attempts.   b. Plot the distribution and describe the shape}}\label{a.-calculate-the-probability-of-having-12-successes-in-20-attempts.-b.-plot-the-distribution-and-describe-the-shape}}

    \begin{Verbatim}[commandchars=\\\{\}]
{\color{incolor}In [{\color{incolor} }]:} a. P\PY{o}{=}\PY{l+m}{20}C12\PY{o}{*}\PY{p}{(}\PY{l+m}{0.7}\PY{p}{)}\PY{o}{\PYZca{}}\PY{l+m}{12}\PY{o}{*}\PY{p}{(}\PY{l+m}{0.3}\PY{p}{)}\PY{o}{\PYZca{}}\PY{l+m}{8} \PY{o}{=}\PY{l+m}{0.114396739704861}
\end{Verbatim}


    \begin{Verbatim}[commandchars=\\\{\}]
{\color{incolor}In [{\color{incolor}52}]:} dbinom\PY{p}{(}\PY{l+m}{12}\PY{p}{,}\PY{l+m}{20}\PY{p}{,}\PY{l+m}{0.7}\PY{p}{)}
\end{Verbatim}


    0.114396739704861

    
    \begin{Verbatim}[commandchars=\\\{\}]
{\color{incolor}In [{\color{incolor} }]:} b.
\end{Verbatim}


    \begin{Verbatim}[commandchars=\\\{\}]
{\color{incolor}In [{\color{incolor}54}]:} pmf \PY{o}{\PYZlt{}\PYZhy{}} dbinom\PY{p}{(}\PY{l+m}{0}\PY{o}{:}\PY{l+m}{19}\PY{p}{,} size \PY{o}{=} \PY{l+m}{20}\PY{p}{,} prob \PY{o}{=} \PY{l+m}{0.7}\PY{p}{)}
         plot\PY{p}{(}pmf\PY{p}{,} type \PY{o}{=} \PY{l+s}{\PYZdq{}}\PY{l+s}{h\PYZdq{}}\PY{p}{,} ylim\PY{o}{=} \PY{k+kt}{c}\PY{p}{(}\PY{l+m}{0}\PY{p}{,}\PY{l+m}{0.5}\PY{p}{)}\PY{p}{)}
         points\PY{p}{(}pmf\PY{p}{,}pch\PY{o}{=}\PY{l+m}{19}\PY{p}{)}
\end{Verbatim}


    \begin{center}
    \adjustimage{max size={0.9\linewidth}{0.9\paperheight}}{output_30_0.png}
    \end{center}
    { \hspace*{\fill} \\}
    
    \hypertarget{a-study-has-shown-that-10-in-250-people-are-infected-with-a-common-cold-virus-however-the-gold-standard-tests-although-accurate-are-not-100-perfect-where-in-fact-if-a-person-has-the-virus-the-probability-of-testing-positive-is-90.}{%
\subsubsection{7. A study has shown that 10 in 250 people are infected
with a common cold virus, however, the gold standard tests although
accurate are not 100\% perfect, where in fact if a person has the virus
the probability of testing positive is
90\%.}\label{a-study-has-shown-that-10-in-250-people-are-infected-with-a-common-cold-virus-however-the-gold-standard-tests-although-accurate-are-not-100-perfect-where-in-fact-if-a-person-has-the-virus-the-probability-of-testing-positive-is-90.}}

\hypertarget{what-the-the-probability-that-a-person-chosen-at-random-has-the-virus-and-tests-positive}{%
\paragraph{\texorpdfstring{ What the the probability that a person
chosen at random has the virus and tests
positive?}{   What the the probability that a person chosen at random has the virus and tests positive?}}\label{what-the-the-probability-that-a-person-chosen-at-random-has-the-virus-and-tests-positive}}

    \begin{Verbatim}[commandchars=\\\{\}]
{\color{incolor}In [{\color{incolor} }]:} P\PY{p}{(}correct\PY{o}{|}virus\PY{p}{)}\PY{o}{=}\PY{l+m}{0.9}\PY{p}{,} P\PY{p}{(}virus\PY{p}{)}\PY{o}{=}\PY{l+m}{1}\PY{o}{/}\PY{l+m}{25}\PY{p}{,} P\PY{p}{(}correct and virus\PY{p}{)}\PY{o}{=}P\PY{p}{(}virus\PY{p}{)}\PY{o}{*}P\PY{p}{(}correct\PY{o}{|}virus\PY{p}{)}\PY{o}{=}\PY{l+m}{0.036}
\end{Verbatim}


    \hypertarget{in-an-italian-gambling-game-a-win-is-when-i-get-at-least-11-when-three-six-sided-dice-are-thrown.-run-a-100000-trial-simulation-of-the-above-game-to-answer-the-following-questions}{%
\subsubsection{8. In an Italian gambling game, a win is when I get at
least 11 when three six-sided dice are thrown. Run a 100000 trial
simulation of the above game to answer the following
questions:}\label{in-an-italian-gambling-game-a-win-is-when-i-get-at-least-11-when-three-six-sided-dice-are-thrown.-run-a-100000-trial-simulation-of-the-above-game-to-answer-the-following-questions}}

 1. Would I, in the long run win the game? 2. Which is more likely when
throwing three dice: an 11 or a 12? 3. What is the probability of
getting a sum no greater than 7~or~no less then 15 when throwing three
dice 

    \begin{Verbatim}[commandchars=\\\{\}]
{\color{incolor}In [{\color{incolor}75}]:} dice\PYZus{}sample \PY{o}{=} \PY{k+kr}{function}\PY{p}{(}\PY{p}{)}\PY{p}{\PYZob{}}\PY{k+kp}{sum}\PY{p}{(}\PY{k+kp}{sample}\PY{p}{(}\PY{l+m}{1}\PY{o}{:}\PY{l+m}{6}\PY{p}{,}\PY{l+m}{3}\PY{p}{,}replace \PY{o}{=} \PY{n+nb+bp}{T}\PY{p}{)}\PY{p}{)}\PY{p}{\PYZcb{}}
\end{Verbatim}


    \begin{Verbatim}[commandchars=\\\{\}]
{\color{incolor}In [{\color{incolor}76}]:} a \PY{o}{=} \PY{k+kp}{replicate}\PY{p}{(}\PY{l+m}{100000}\PY{p}{,} \PY{k+kp}{ifelse}\PY{p}{(}dice\PYZus{}sample\PY{p}{(}\PY{p}{)}\PY{o}{\PYZgt{}=} \PY{l+m}{11}\PY{p}{,}\PY{l+m}{1}\PY{p}{,}\PY{l+m}{0}\PY{p}{)}\PY{p}{)}
         \PY{c+c1}{\PYZsh{}a                    }
         \PY{k+kp}{sum}\PY{p}{(}a\PY{p}{)}\PY{o}{/}\PY{l+m}{100000}
\end{Verbatim}


    0.49964

    
    \begin{Verbatim}[commandchars=\\\{\}]
{\color{incolor}In [{\color{incolor} }]:} \PY{l+m}{1}\PY{l+m}{.} Not sure. The probability to win is nearly \PY{l+m}{0.5}\PY{l+m}{.}
\end{Verbatim}


    \begin{Verbatim}[commandchars=\\\{\}]
{\color{incolor}In [{\color{incolor}77}]:} a \PY{o}{=} \PY{k+kp}{replicate}\PY{p}{(}\PY{l+m}{100000}\PY{p}{,} \PY{k+kp}{ifelse}\PY{p}{(}dice\PYZus{}sample\PY{p}{(}\PY{p}{)}\PY{o}{==} \PY{l+m}{11}\PY{p}{,}\PY{l+m}{1}\PY{p}{,}\PY{l+m}{0}\PY{p}{)}\PY{p}{)}
         \PY{c+c1}{\PYZsh{}a                    }
         \PY{k+kp}{sum}\PY{p}{(}a\PY{p}{)}\PY{o}{/}\PY{l+m}{100000}
\end{Verbatim}


    0.12536

    
    \begin{Verbatim}[commandchars=\\\{\}]
{\color{incolor}In [{\color{incolor}78}]:} a \PY{o}{=} \PY{k+kp}{replicate}\PY{p}{(}\PY{l+m}{100000}\PY{p}{,} \PY{k+kp}{ifelse}\PY{p}{(}dice\PYZus{}sample\PY{p}{(}\PY{p}{)}\PY{o}{==} \PY{l+m}{12}\PY{p}{,}\PY{l+m}{1}\PY{p}{,}\PY{l+m}{0}\PY{p}{)}\PY{p}{)}
         \PY{c+c1}{\PYZsh{}a                    }
         \PY{k+kp}{sum}\PY{p}{(}a\PY{p}{)}\PY{o}{/}\PY{l+m}{100000}
\end{Verbatim}


    0.11479

    
    \begin{Verbatim}[commandchars=\\\{\}]
{\color{incolor}In [{\color{incolor} }]:} \PY{l+m}{2}\PY{l+m}{.} P\PY{p}{(}sum\PY{o}{=}\PY{l+m}{11}\PY{p}{)}\PY{o}{=}\PY{l+m}{0.12536}
        P\PY{p}{(}sum\PY{o}{=}\PY{l+m}{12}\PY{p}{)}\PY{o}{=}\PY{l+m}{0.11479}
        \PY{l+m}{12} is more likely to throw \PY{l+m}{3} dices
\end{Verbatim}


    \hypertarget{in-a-company-34-of-the-females-are-single}{%
\subsubsection{9. In a company 3/4 of the females are
single,}\label{in-a-company-34-of-the-females-are-single}}

\hypertarget{calculate-the-probability-that-within-the-first-5-randomly-selected-females-we-find-the-first-single-woman-in-average-in-how-many-people-we-need-to-select-before-find-a-single-female}{%
\subsubsection{\texorpdfstring{ Calculate the probability that within
the first 5 randomly selected females we find the first single woman? In
average in how many people we need to select before find a single
female?}{    Calculate the probability that within the first 5 randomly selected females we find the first single woman?    In average in how many people we need to select before find a single female?}}\label{calculate-the-probability-that-within-the-first-5-randomly-selected-females-we-find-the-first-single-woman-in-average-in-how-many-people-we-need-to-select-before-find-a-single-female}}

    \begin{Verbatim}[commandchars=\\\{\}]
{\color{incolor}In [{\color{incolor} }]:} \PY{l+m}{1}\PY{l+m}{.} P \PY{o}{=} \PY{l+m}{3}\PY{o}{/}\PY{l+m}{4}\PY{o}{*}\PY{p}{(}\PY{l+m}{1}\PY{o}{/}\PY{l+m}{4}\PY{p}{)}\PY{o}{*}\PY{p}{(}\PY{l+m}{3}\PY{o}{/}\PY{l+m}{4}\PY{p}{)}\PY{o}{+}\PY{p}{(}\PY{l+m}{1}\PY{o}{/}\PY{l+m}{4}\PY{p}{)}\PY{o}{\PYZca{}}\PY{l+m}{2}\PY{o}{*}\PY{l+m}{3}\PY{o}{/}\PY{l+m}{4}\PY{o}{+}\PY{p}{(}\PY{l+m}{1}\PY{o}{/}\PY{l+m}{4}\PY{p}{)}\PY{o}{\PYZca{}}\PY{l+m}{3}\PY{o}{*}\PY{l+m}{3}\PY{o}{/}\PY{l+m}{4}\PY{o}{+}\PY{p}{(}\PY{l+m}{1}\PY{o}{/}\PY{l+m}{4}\PY{p}{)}\PY{o}{\PYZca{}}\PY{l+m}{4}\PY{o}{*}\PY{l+m}{3}\PY{o}{/}\PY{l+m}{4}\PY{o}{=}\PY{l+m}{0.2}
        \PY{l+m}{2}\PY{l+m}{.} \PY{l+m}{1}\PY{o}{/}P \PY{o}{=} \PY{l+m}{5}
\end{Verbatim}


    \hypertarget{lets-use-a-mouse-random-walk-the-closed-maze-where-a-mouse-always-start-on-the-first-chamber-and-can-move-randomly-to-different-chambers-until-it-finds-a-cheese-in-chambers-7-or-9.-from-the-following-diagram-calculate}{%
\subsubsection{10. Lets use a mouse random walk The Closed Maze, where a
mouse always start on the first chamber and can move randomly to
different chambers until it finds a cheese in chambers 7 or 9. From the
following diagram
calculate:}\label{lets-use-a-mouse-random-walk-the-closed-maze-where-a-mouse-always-start-on-the-first-chamber-and-can-move-randomly-to-different-chambers-until-it-finds-a-cheese-in-chambers-7-or-9.-from-the-following-diagram-calculate}}

\begin{figure}
\centering
\includegraphics{Mouse_random.png}
\caption{title}
\end{figure}

 1. The transition matrix 2. Write a function that simulates this random
walk (5000 times ) the mouse starts always from the 1st chamber, 3. Plot
the mouse random walk simulation using \textbf{ONE} of the following
vector (steps - N) sizes (10,15, 50,100), 4. what are the probabilities
of finishing in each chamber at each one of these steps sizes? (table of
4 rows (vector size -N) vs 9 columns (chambers))

    \begin{Verbatim}[commandchars=\\\{\}]
{\color{incolor}In [{\color{incolor}85}]:} \PY{k+kn}{library}\PY{p}{(}markovchain\PY{p}{)}
         
         P\PY{o}{=}\PY{k+kt}{matrix}\PY{p}{(}\PY{l+m}{0}\PY{p}{,}\PY{l+m}{9}\PY{p}{,}\PY{l+m}{9}\PY{p}{)}
         
         P\PY{p}{[}\PY{l+m}{1}\PY{p}{,}\PY{p}{]}\PY{o}{=}\PY{k+kt}{c}\PY{p}{(}\PY{l+m}{0}\PY{p}{,} \PY{l+m}{0.5}\PY{p}{,} \PY{l+m}{0}\PY{p}{,} \PY{l+m}{0.5}\PY{p}{,} \PY{l+m}{0}\PY{p}{,} \PY{l+m}{0}\PY{p}{,} \PY{l+m}{0}\PY{p}{,} \PY{l+m}{0}\PY{p}{,} \PY{l+m}{0}\PY{p}{)}
         P\PY{p}{[}\PY{l+m}{2}\PY{p}{,}\PY{p}{]}\PY{o}{=}\PY{k+kt}{c}\PY{p}{(}\PY{l+m}{1}\PY{o}{/}\PY{l+m}{3}\PY{p}{,} \PY{l+m}{0}\PY{p}{,} \PY{l+m}{1}\PY{o}{/}\PY{l+m}{3}\PY{p}{,} \PY{l+m}{0}\PY{p}{,} \PY{l+m}{1}\PY{o}{/}\PY{l+m}{3}\PY{p}{,}\PY{l+m}{0}\PY{p}{,} \PY{l+m}{0}\PY{p}{,} \PY{l+m}{0}\PY{p}{,} \PY{l+m}{0}\PY{p}{)}
         P\PY{p}{[}\PY{l+m}{3}\PY{p}{,}\PY{p}{]}\PY{o}{=}\PY{k+kt}{c}\PY{p}{(}\PY{l+m}{0}\PY{p}{,} \PY{l+m}{1}\PY{o}{/}\PY{l+m}{2}\PY{p}{,} \PY{l+m}{0}\PY{p}{,} \PY{l+m}{0}\PY{p}{,} \PY{l+m}{0}\PY{p}{,} \PY{l+m}{1}\PY{o}{/}\PY{l+m}{2}\PY{p}{,} \PY{l+m}{0}\PY{p}{,} \PY{l+m}{0}\PY{p}{,} \PY{l+m}{0}\PY{p}{)}
         P\PY{p}{[}\PY{l+m}{4}\PY{p}{,}\PY{p}{]}\PY{o}{=}\PY{k+kt}{c}\PY{p}{(}\PY{l+m}{1}\PY{o}{/}\PY{l+m}{3}\PY{p}{,} \PY{l+m}{0}\PY{p}{,} \PY{l+m}{0}\PY{p}{,} \PY{l+m}{0}\PY{p}{,} \PY{l+m}{1}\PY{o}{/}\PY{l+m}{3}\PY{p}{,} \PY{l+m}{0}\PY{p}{,}\PY{l+m}{1}\PY{o}{/}\PY{l+m}{3}\PY{p}{,} \PY{l+m}{0}\PY{p}{,} \PY{l+m}{0}\PY{p}{)}
         P\PY{p}{[}\PY{l+m}{5}\PY{p}{,}\PY{p}{]}\PY{o}{=}\PY{k+kt}{c}\PY{p}{(}\PY{l+m}{0}\PY{p}{,} \PY{l+m}{1}\PY{o}{/}\PY{l+m}{4}\PY{p}{,} \PY{l+m}{0}\PY{p}{,}\PY{l+m}{1}\PY{o}{/}\PY{l+m}{4}\PY{p}{,} \PY{l+m}{0}\PY{p}{,}\PY{l+m}{1}\PY{o}{/}\PY{l+m}{4}\PY{p}{,} \PY{l+m}{0}\PY{p}{,}\PY{l+m}{1}\PY{o}{/}\PY{l+m}{4}\PY{p}{,}\PY{l+m}{0}\PY{p}{)}
         P\PY{p}{[}\PY{l+m}{6}\PY{p}{,}\PY{p}{]}\PY{o}{=}\PY{k+kt}{c}\PY{p}{(}\PY{l+m}{0}\PY{p}{,} \PY{l+m}{0}\PY{p}{,} \PY{l+m}{1}\PY{o}{/}\PY{l+m}{3}\PY{p}{,} \PY{l+m}{0}\PY{p}{,} \PY{l+m}{1}\PY{o}{/}\PY{l+m}{3}\PY{p}{,} \PY{l+m}{0}\PY{p}{,} \PY{l+m}{0}\PY{p}{,} \PY{l+m}{0}\PY{p}{,} \PY{l+m}{1}\PY{o}{/}\PY{l+m}{3}\PY{p}{)}
         P\PY{p}{[}\PY{l+m}{7}\PY{p}{,}\PY{p}{]}\PY{o}{=}\PY{k+kt}{c}\PY{p}{(}\PY{l+m}{0}\PY{p}{,} \PY{l+m}{0}\PY{p}{,} \PY{l+m}{0}\PY{p}{,} \PY{l+m}{0}\PY{p}{,} \PY{l+m}{0}\PY{p}{,} \PY{l+m}{0}\PY{p}{,} \PY{l+m}{1}\PY{p}{,} \PY{l+m}{0}\PY{p}{,} \PY{l+m}{0}\PY{p}{)}
         P\PY{p}{[}\PY{l+m}{8}\PY{p}{,}\PY{p}{]}\PY{o}{=}\PY{k+kt}{c}\PY{p}{(}\PY{l+m}{0}\PY{p}{,} \PY{l+m}{0}\PY{p}{,} \PY{l+m}{0}\PY{p}{,} \PY{l+m}{0}\PY{p}{,}\PY{l+m}{1}\PY{o}{/}\PY{l+m}{3}\PY{p}{,}\PY{l+m}{0}\PY{p}{,}\PY{l+m}{1}\PY{o}{/}\PY{l+m}{3}\PY{p}{,}\PY{l+m}{0}\PY{p}{,}\PY{l+m}{1}\PY{o}{/}\PY{l+m}{3}\PY{p}{)}
         P\PY{p}{[}\PY{l+m}{9}\PY{p}{,}\PY{p}{]}\PY{o}{=}\PY{k+kt}{c}\PY{p}{(}\PY{l+m}{0}\PY{p}{,} \PY{l+m}{0}\PY{p}{,} \PY{l+m}{0}\PY{p}{,} \PY{l+m}{0}\PY{p}{,} \PY{l+m}{0}\PY{p}{,} \PY{l+m}{0}\PY{p}{,} \PY{l+m}{0}\PY{p}{,} \PY{l+m}{0}\PY{p}{,} \PY{l+m}{1}\PY{p}{)}
         
         P
\end{Verbatim}


    \begin{tabular}{lllllllll}
	 0.0000000 & 0.50      & 0.0000000 & 0.50      & 0.0000000 & 0.00      & 0.0000000 & 0.00      & 0.0000000\\
	 0.3333333 & 0.00      & 0.3333333 & 0.00      & 0.3333333 & 0.00      & 0.0000000 & 0.00      & 0.0000000\\
	 0.0000000 & 0.50      & 0.0000000 & 0.00      & 0.0000000 & 0.50      & 0.0000000 & 0.00      & 0.0000000\\
	 0.3333333 & 0.00      & 0.0000000 & 0.00      & 0.3333333 & 0.00      & 0.3333333 & 0.00      & 0.0000000\\
	 0.0000000 & 0.25      & 0.0000000 & 0.25      & 0.0000000 & 0.25      & 0.0000000 & 0.25      & 0.0000000\\
	 0.0000000 & 0.00      & 0.3333333 & 0.00      & 0.3333333 & 0.00      & 0.0000000 & 0.00      & 0.3333333\\
	 0.0000000 & 0.00      & 0.0000000 & 0.00      & 0.0000000 & 0.00      & 1.0000000 & 0.00      & 0.0000000\\
	 0.0000000 & 0.00      & 0.0000000 & 0.00      & 0.3333333 & 0.00      & 0.3333333 & 0.00      & 0.3333333\\
	 0.0000000 & 0.00      & 0.0000000 & 0.00      & 0.0000000 & 0.00      & 0.0000000 & 0.00      & 1.0000000\\
\end{tabular}


    
    \begin{Verbatim}[commandchars=\\\{\}]
{\color{incolor}In [{\color{incolor}137}]:} Markov2 \PY{o}{=} \PY{k+kr}{function}\PY{p}{(}N\PY{p}{,} Pi0\PY{p}{,} P\PY{p}{)}\PY{p}{\PYZob{}} \PY{c+c1}{\PYZsh{}N = number of steps, N0 = initial probs, P matrix}
            P0 \PY{o}{=} \PY{k+kt}{c}\PY{p}{(}\PY{l+m}{0.0005}\PY{p}{,}\PY{l+m}{0.0005}\PY{p}{,}\PY{l+m}{0.199}\PY{p}{,}\PY{l+m}{0.4}\PY{p}{,}\PY{l+m}{0.0005}\PY{p}{,}\PY{l+m}{0.0005}\PY{p}{,}\PY{l+m}{0.0005}\PY{p}{,}\PY{l+m}{0.0005}\PY{p}{,}\PY{l+m}{0.0005}\PY{p}{)}
            P \PY{o}{=} P\PY{o}{*}P0
            X\PY{o}{=}\PY{k+kt}{matrix}\PY{p}{(}\PY{l+m}{0}\PY{p}{,}\PY{l+m}{1}\PY{p}{,}N\PY{p}{)}
            a \PY{o}{=} \PY{l+m}{1} \PY{c+c1}{\PYZsh{}\PYZsh{}Start the random walk in position 1}
            X\PY{p}{[}\PY{l+m}{1}\PY{p}{]}\PY{o}{=}a
            \PY{k+kr}{for} \PY{p}{(}i \PY{k+kr}{in} \PY{l+m}{2}\PY{o}{:}N\PY{p}{)} \PY{p}{\PYZob{}}
              a\PY{o}{=}\PY{k+kp}{sample}\PY{p}{(}\PY{k+kt}{c}\PY{p}{(}\PY{l+m}{1}\PY{o}{:}\PY{l+m}{9}\PY{p}{)}\PY{p}{,}\PY{l+m}{1}\PY{p}{,}replace\PY{o}{=}\PY{n+nb+bp}{T}\PY{p}{,} P\PY{p}{[}a\PY{p}{,}\PY{p}{]}\PY{p}{)}
              X\PY{p}{[}i\PY{p}{]}\PY{o}{=}a
            \PY{p}{\PYZcb{}}
            b \PY{o}{=} \PY{k+kp}{as.vector}\PY{p}{(}X\PY{p}{)}
            \PY{k+kr}{return}\PY{p}{(}b\PY{p}{)}
          \PY{p}{\PYZcb{}}
          P0 \PY{o}{=} \PY{k+kt}{c}\PY{p}{(}\PY{l+m}{0.0005}\PY{p}{,}\PY{l+m}{0.0005}\PY{p}{,}\PY{l+m}{0.199}\PY{p}{,}\PY{l+m}{0.4}\PY{p}{,}\PY{l+m}{0.0005}\PY{p}{,}\PY{l+m}{0.0005}\PY{p}{,}\PY{l+m}{0.0005}\PY{p}{,}\PY{l+m}{0.0005}\PY{p}{,}\PY{l+m}{0.0005}\PY{p}{)}
          \PY{c+c1}{\PYZsh{}Markov2(10,P0,P)}
          N \PY{o}{=}\PY{l+m}{10} \PY{c+c1}{\PYZsh{} Use 10 steps}
          plot\PY{p}{(}\PY{k+kc}{NA}\PY{p}{,} xlim\PY{o}{=}\PY{k+kt}{c}\PY{p}{(}\PY{l+m}{0}\PY{p}{,}N\PY{p}{)}\PY{p}{,} ylim\PY{o}{=}\PY{k+kt}{c}\PY{p}{(}\PY{l+m}{0}\PY{p}{,}\PY{l+m}{10}\PY{p}{)}\PY{p}{)}\PY{c+c1}{\PYZsh{}empty plot}
          datas \PY{o}{=} \PY{k+kt}{matrix}\PY{p}{(}ncol \PY{o}{=} N\PY{p}{,} nrow \PY{o}{=} \PY{l+m}{5000}\PY{p}{)}
          \PY{k+kr}{for} \PY{p}{(}i \PY{k+kr}{in} \PY{l+m}{1}\PY{o}{:}\PY{l+m}{5000}\PY{p}{)}\PY{p}{\PYZob{}}
            datas\PY{p}{[}i\PY{p}{,}\PY{p}{]} \PY{o}{=} Markov2\PY{p}{(}N\PY{p}{,}P0\PY{p}{,}P\PY{p}{)}
            condir \PY{o}{=} datas\PY{p}{[}i\PY{p}{,}\PY{p}{]}
            col \PY{o}{=} \PY{p}{(}condir\PY{p}{[}\PY{l+m}{10}\PY{p}{]}\PY{o}{==}\PY{l+m}{7} \PY{o}{|} condir\PY{p}{[}\PY{l+m}{10}\PY{p}{]}\PY{o}{==}\PY{l+m}{9}\PY{p}{)}
            lines\PY{p}{(}condir\PY{p}{,} lwd\PY{o}{=}\PY{l+m}{2}\PY{p}{,}col \PY{o}{=} \PY{k+kp}{ifelse}\PY{p}{(}\PY{k+kp}{col}\PY{p}{,} \PY{l+s}{\PYZdq{}}\PY{l+s}{coral\PYZdq{}}\PY{p}{,}\PY{l+s}{\PYZdq{}}\PY{l+s}{forestgreen\PYZdq{}}\PY{p}{)}\PY{p}{)}
          \PY{p}{\PYZcb{}}
          \PY{c+c1}{\PYZsh{} length(datas)}
          \PY{c+c1}{\PYZsh{} datas}
          \PY{c+c1}{\PYZsh{} datas[45001:50000]}
\end{Verbatim}


    \begin{center}
    \adjustimage{max size={0.9\linewidth}{0.9\paperheight}}{output_44_0.png}
    \end{center}
    { \hspace*{\fill} \\}
    
    \begin{Verbatim}[commandchars=\\\{\}]
{\color{incolor}In [{\color{incolor}136}]:} \PY{k+kp}{table}\PY{p}{(}datas\PY{p}{[}\PY{l+m}{45001}\PY{o}{:}\PY{l+m}{50000}\PY{p}{]}\PY{p}{)}
\end{Verbatim}


    
    \begin{verbatim}

   2    4    6    7    8    9 
 681  475  471 2083  214 1076 
    \end{verbatim}

    
    \begin{Verbatim}[commandchars=\\\{\}]
{\color{incolor}In [{\color{incolor} }]:} what are the probabilities of finishing \PY{k+kr}{in} each chamber at each one of these steps sizes\PY{o}{?} \PY{p}{(}table of \PY{l+m}{4} rows \PY{p}{(}\PY{k+kt}{vector} size \PY{o}{\PYZhy{}}N\PY{p}{)} vs \PY{l+m}{9} columns \PY{p}{(}chambers\PY{p}{)}\PY{p}{)}
        
        step size\PY{o}{:} \PY{p}{(}\PY{l+m}{10}\PY{p}{,}\PY{l+m}{15}\PY{p}{,} \PY{l+m}{50}\PY{p}{,}\PY{l+m}{100}\PY{p}{)}
        
        step \PY{l+m}{1}  \PY{l+m}{2}    \PY{l+m}{4}  \PY{l+m}{5}  \PY{l+m}{6}    \PY{l+m}{7}    \PY{l+m}{8}    \PY{l+m}{9} 
        \PY{l+m}{10}   \PY{l+m}{0} \PY{l+m}{717}  \PY{l+m}{448} \PY{l+m}{0} \PY{l+m}{430} \PY{l+m}{2107}  \PY{l+m}{231} \PY{l+m}{1067} 
        \PY{l+m}{15}   \PY{l+m}{0} \PY{l+m}{725}  \PY{l+m}{450} \PY{l+m}{0} \PY{l+m}{455} \PY{l+m}{2076}  \PY{l+m}{194} \PY{l+m}{1100} 
        \PY{l+m}{50}   \PY{l+m}{0} \PY{l+m}{732}  \PY{l+m}{463} \PY{l+m}{0} \PY{l+m}{505} \PY{l+m}{2069}  \PY{l+m}{214} \PY{l+m}{1017} 
        \PY{l+m}{100}  \PY{l+m}{0} \PY{l+m}{681}  \PY{l+m}{475} \PY{l+m}{0} \PY{l+m}{471} \PY{l+m}{2083}  \PY{l+m}{214} \PY{l+m}{1076}
        
        step \PY{l+m}{1}  \PY{l+m}{2}     \PY{l+m}{4}    \PY{l+m}{5}  \PY{l+m}{6}       \PY{l+m}{7}     \PY{l+m}{8}     \PY{l+m}{9} 
        \PY{l+m}{10}   \PY{l+m}{0} \PY{l+m}{0.14}  \PY{l+m}{0.09}  \PY{l+m}{0} \PY{l+m}{0.086}   \PY{l+m}{0.42}  \PY{l+m}{0.046} \PY{l+m}{0.21} 
        \PY{l+m}{15}   \PY{l+m}{0} \PY{l+m}{0.15}  \PY{l+m}{0.09}  \PY{l+m}{0} \PY{l+m}{0.091}   \PY{l+m}{0.4}   \PY{l+m}{0.388} \PY{l+m}{0.22} 
        \PY{l+m}{50}   \PY{l+m}{0} \PY{l+m}{0.15}  \PY{l+m}{0.09}  \PY{l+m}{0} \PY{l+m}{0.11}    \PY{l+m}{0.4}   \PY{l+m}{0.428} \PY{l+m}{0.2} 
        \PY{l+m}{100}  \PY{l+m}{0} \PY{l+m}{0.14}  \PY{l+m}{0.095} \PY{l+m}{0} \PY{l+m}{0.942}   \PY{l+m}{0.41}  \PY{l+m}{0.428} \PY{l+m}{0.2}
\end{Verbatim}



    % Add a bibliography block to the postdoc
    
    
    
    \end{document}
